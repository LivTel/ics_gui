\documentclass[10pt,a4paper]{article}
\pagestyle{plain}
\textwidth 16cm
\textheight 21cm
\oddsidemargin -0.5cm
\topmargin 0cm

\title{CCD Control System : Basic User Guide}
\author{C. J. Mottram}
\date{}
\begin{document}
\pagenumbering{arabic}
\thispagestyle{empty}
\maketitle
\begin{abstract}
This document describes the basic operations needed to control the CCD Control System, manually,
via the Instrument Control System GUI.
\end{abstract}

\centerline{\Large History}
\begin{center}
\begin{tabular}{|l|l|l|p{15em}|}
\hline
{\bf Version} & {\bf Author} & {\bf Date} & {\bf Notes} \\
\hline
0.1 & C. J. Mottram & 07/08/03 & First draft. \\
0.2 & C. J. Mottram & 13/08/03 & More diagrams. \\
\hline
\end{tabular}
\end{center}

\newpage
\tableofcontents
\listoffigures
\listoftables
\newpage

\section{Introduction}
The CCS is the CCD Control System, which allows control of an SDSU CCD controller, and associated filter wheel.
In normal operation the CCS is controlled from the RCS (Robotic Control System), to provide the ability
to control the instrument robotically. However, for engineering operations, the CCD Control System can be controlled
manually from an instrument graphical user interface called IcsGUI (formerly known as CcsGUI). It is also possible
to control the system without the CCD Control System running, using an interface called voodoo (supplied by SDSU),
or by using very low level command line programs.

This document is mainly concerned with how to boot the CCS and IcsGUI, and to control the camera from the IcsGUI.

\section{Connection of system components}
Refer to appropriate TTL documentation.

\section{Power on Sequence}
\subsection{Usual sequence}
The correct power on sequence is as follows:

\begin{itemize}
\item Ensure all connections are connected.
\item Power on the SDSU controller box (grey box). See Section \ref{sec:diagramsdsupsu}.
\item There should be 2 green lights come on on the back of the controller (plus one half lit). 
      See Section \ref{sec:diagramdewarelectronicsbox}.
      If a red light comes on (second from the right) , the chances are the SDSU controller has a temperature fault.
      This means the controller has got too hot, causing a thermistor in the controller to trip.
\item Power on the filter wheel power supply.
\item Power on the control computer. See Section \ref{sec:diagramsunultra5back}.
\end{itemize}

\subsection{Cryotiger}
Note the Cryotiger power can be switched on independently of the rest of the power, it should be turned on
at least six hours before you want to observe (to cool the dewar down).

\subsection{Instrument Control Computer reboot}
If the instrument computer is currently running, don't panic! The chances are the automatic boot-up sequence will have
timed out as there was no power to the controller. Just log into the control computer as {\bf root} 
(see Section \ref{sec:instrumentcontrolcomputer}), 
and type {\em reboot} to restart the automatic boot-up sequence.

\subsection{Check fibre optics connected}
If all goes well seven green lights should come on on the SDSU controller and the filter wheels should home themselves.
If the lights don't come on, check the green light on the PCI card plugged into the back of the control computer.
If it is not on, then the fibre optic communication between the PCI card and the SDSU controller is not working.
Try swapping the two connectors round at one end, if this doesn't work check the fibre.

More advanced diagnostics can be found in the Advanced User Guide \cite{bib:ccs_aug}.

\subsection{Control Computer locks up}
In very rare cases the power up sequence may cause the Instrument Control Computer to appear to 'freeze' or 'lock up'.
If this is the case, There is a small white button on the back of the PCI card that should be pressed (one or
more times if necessary). You may have to re-start the power up sequence in this case (i.e. power down everything,
then power up everything properly). This problem is usually caused by incorrect power up sequences.

\section{Instrument Control Computer}
\label{sec:instrumentcontrolcomputer}

The instrument support or control computer is usually a Sun workstation, although a Linux box can be used as well.
All instrument control computers are supplied with two accounts, as documented in Table \ref{tab:iscaccounts}.

\begin{table}[!h]
\begin{center}
\begin{tabular}{|l|l|p{10em}|}
\hline
{\bf User-name} & {\bf Password} 	& {\bf Description} \\ \hline
root 		& ng@tr00t 		& System Administrator \\ \hline
eng 		& ng@teng 		& Engineer User \\ \hline
\end{tabular}
\end{center}
\caption{\em Accounts on the instrument control computer.}
\label{tab:iscaccounts}
\end{table}

For normal operations the {\bf eng} user should be used. {\bf root} allows you to change the operating system
setup, and other tasks like going into low-level engineering mode (stopping the CCD Control System
robotic software).

See the Advanced User Guide \cite{bib:ccs_aug} for more details.

\section{Starting the Instrument Control Graphical User Interface}
\label{sec:starticsgui}

To start the instrument control graphical user interface, following tasks must be performed:

\begin{itemize}
\item You should log into the control computer as user {\bf eng}. See Section \ref{sec:instrumentcontrolcomputer}.
\item You should bring up a terminal. Under Linux this is easy, under Solaris it is a little confusing.
	Under Solaris, right click on the desktop background, select {\bf Tools} on the workspace menu, and select
	{\bf Terminal} on the Tools menu. See Figure \ref{fig:startterminalsolaris}.
\item In the terminal, type the following command {\bf icsgui}, and press return.
\item The IcsGUI main screen should then come up as shown in Figure \ref{fig:icsguimainscreen}.
\end{itemize}

\setlength{\unitlength}{1in}
\begin{figure}[!h]
	\begin{center}
		\begin{picture}(8.0,5.0)(0.0,0.0)
			\put(0,0){\special{psfile=terminal_low_res.eps hscale=40 vscale=40}}
		\end{picture}
	\end{center}
	\caption{\em Starting a terminal on Solaris.}
	\label{fig:startterminalsolaris} 
\end{figure}

\setlength{\unitlength}{1in}
\begin{figure}[!h]
	\begin{center}
		\begin{picture}(3.0,3.0)(0.0,0.0)
			\put(0,0){\special{psfile=icsgui_main_screen.eps hscale=50 vscale=50}}
		\end{picture}
	\end{center}
	\caption{\em The IcsGUI main screen.}
	\label{fig:icsguimainscreen} 
\end{figure}


\section{Configuring the CCS to talk to the IcsGUI}
\label{sec:issconfiguration}
Before using all the features of the IcsGUI, you need to decide how the CCS is going to interact with it and the RCS.

Normally the CCS sends commands to the RCS whilst it is itself processing commands (from the RCS or IcsGUI).
The following is an example list of commands the CCS sends:

\begin{itemize}
\item AG\_START
\item AG\_STOP
\item GET\_FITS 
\item MOVE\_FOLD
\item OFFSET\_FOCUS
\item OFFSET\_RA\_DEC
\item OFFSET\_ROTATOR
\item SET\_FOCUS
\end{itemize}

If these commands are sent to the RCS, then they are actually performed, or in the case of GET\_FITS the
headers returned contain the real data from the telescope.

However, the IcsGUI has the capability of faking these commands, so that if the RCS is not running you can still
use the camera. 

Figure \ref{fig:rcsicsguicomms} shows the two possible interactions. In both cases a MULTRUN is sent to the CCS, 
which causes the CCS to issue a {\bf MOVE\_FOLD} command. 
On the left it is issued to the RCS, which sends the relevant command
to the TCS. On the right the IcsGUI handles the {\bf MOVE\_FOLD}, and 'pretends' to do the command, or displays
a dialog for the user to do it interactively.

\setlength{\unitlength}{1in}
\begin{figure}[!h]
	\begin{center}
		\begin{picture}(3.0,3.0)(0.0,0.0)
			\put(0,0){\special{psfile=rcs_ccs_interaction.eps hscale=50 vscale=50}}
		\end{picture}
	\end{center}
	\caption{\em RCS/IcsGUI communications with the CCS.}
	\label{fig:rcsicsguicomms} 
\end{figure}

To make use of this feature of the IcsGUI, you must first tell the CCS to talk to the IcsGUI rather than
the RCS.

To do this, edit the file {\em /icc/bin/ccs/java/ccs.net.properties}. Change the following property: 
\newline {\bf ccs.net.default\_ISS\_address}. The value after the equals sign should be the name of the machine you
wish to send ISS commands to. (The ISS is nominally part of the RCS, it stands for Instrument Support System).
So either set it to the name of the machine which the RCS is on (as defined in {\em /etc/hosts}) or to the
local machine which the IcsGUI is running on. For more details see \cite{bib:ccs}.

To make the setting take effect, you need to restart the CCS. This should be done by either rebooting the
instrument computer, or sending a level {\em 2} {\bf REBOOT} command using the IcsGUI 
(see Section \ref{sec:rebootcommand}).

\section{Basic operation of IcsGUI}
The main IcsGUI display can be seen in Figure \ref{fig:icsguimainscreen}. The following features may be seen:

\subsection{File Menu}
The file menu is the first menu. It contains an item for displaying a list of threads the IcsGUI is running
(you won't normally use this) and The {\em Exit} button, which terminates IcsGUI.

\subsection{Log Menu}
The Log menu contains one option, a {\em Clear} button which clears the log text area at the bottom of the screen.
If there is a lot of logging on the log text area, the GUI can slow down a bit, so it is a good idea to clear
the logging information once in a while. The logs are also stored in a log file, which is not cleared until the
next IcsGUI is started.

\subsection{Send Menu}
This is the important one, in that it allows access to send all the robotic command set to the CCS. The robotic
command set is divided into a hierarchy, which is reflected in the {\em Send} menu, see Figure \ref{fig:icsguisendmenu}.

\setlength{\unitlength}{1in}
\begin{figure}[!h]
	\begin{center}
		\begin{picture}(3.0,3.0)(0.0,0.0)
			\put(0,0){\special{psfile=icsgui_send_menu.eps hscale=50 vscale=50}}
		\end{picture}
	\end{center}
	\caption{\em The IcsGUI send menu.}
	\label{fig:icsguisendmenu} 
\end{figure}

\subsection{ISS Menu}

The ISS menu allows you to configure how IcsGUI behaves when it is pretending to be the ISS.
The menu is shown in Figure \ref{fig:icsguiissmenu}.

\setlength{\unitlength}{1in}
\begin{figure}[!h]
	\begin{center}
		\begin{picture}(3.0,3.0)(0.0,0.0)
			\put(0,0){\special{psfile=icsgui_iss_menu.eps hscale=50 vscale=50}}
		\end{picture}
	\end{center}
	\caption{\em The IcsGUI ISS menu.}
	\label{fig:icsguiissmenu} 
\end{figure}

The {\em Spoof Requests} option turns on ISS emulation. You need this check-box on to pretend to be the ISS.
The {\em Message Dialog} option, if selected, brings up a message dialog whenever an ISS request is received from
the CCS. This allows you to say, use the TCS to move the science fold when the CCS sends a {\bf MOVE\_FOLD} message.

\subsection{Auto-update}

The {\em Auto-update} check-box is located on the main screen, see Figure \ref{fig:icsguimainscreen}. 
When checked, it will send a robotic
{\bf GET\_STATUS} request to the CCS every {\em n} milliseconds, where {\em n} is specified in the text field
next to the check-box (note {\em n} cannot be less than 1000). The returned status in this used to populate
the {\em status} panel located above the check-box. The level of the robotic {\bf GET\_STATUS} request sent
is determined by the IcsGUI configuration file, it must be at least level {\em 1} to get temperature status.

\subsection{FITS Filename}

The {\em FITS Filename} button is located on the main screen, see Figure \ref{fig:icsguimainscreen}. When
a robotic command issued by the IcsGUI returns a filename (during or at the end of the command sequence) the space
next to the button is filled in with the filename. The {\em FITS Filename} button can then be pressed, and the
relevant FITS image is loaded into the FITS image viewer configured by the IcsGUI config file (this is normally
GAIA).

\section{Configuring for an exposure}
\label{sec:configuring}

{\bf Before taking an image} with the instrument, it must first be configured. This is achieved using the robotic
command	{\bf CONFIG}. To bring up the Config dialog, select the menu option 
{\em Send$\rightarrow$Setup$\rightarrow$Config}, see Figure \ref{fig:icsguisetupmenu}. If an attempt is made to take 
an exposure without configuring the instrument, then the {\bf COMMAND\_DONE} message indicates a failure.

\setlength{\unitlength}{1in}
\begin{figure}[!h]
	\begin{center}
		\begin{picture}(3.0,3.0)(0.0,0.0)
			\put(0,0){\special{psfile=icsgui_setup_menu.eps hscale=50 vscale=50}}
		\end{picture}
	\end{center}
	\caption{\em The IcsGUI setup menu.}
	\label{fig:icsguisetupmenu} 
\end{figure}

\subsection{CONFIG Dialog}

This brings up the CONFIG dialog, see Figure \ref{fig:icsguiconfigdialog}.

\setlength{\unitlength}{1in}
\begin{figure}[!h]
	\begin{center}
		\begin{picture}(2.0,1.0)(0.0,0.0)
			\put(0,0){\special{psfile=icsgui_config_dialog.eps hscale=50 vscale=50}}
		\end{picture}
	\end{center}
	\caption{\em The IcsGUI Config Dialog.}
	\label{fig:icsguiconfigdialog} 
\end{figure}

The Identifier field is automatically filled in by the IcsGUI, it allows the IcsGUI and CCS to identify
specific commands.

The {\em Config} button allows the user to select the configuration of the instrument wanted from
a list. When the configuration has been selected, the blank area to the right of this button is filled in
with the name of the selected configuration. See Section \ref{sec:instrumentconfigurationlist}.

The {\em Ok} button sends the {\bf CONFIG} command to the CCS.

The {\em Cancel} button quits the dialog without sending the {\bf CONFIG} command.

After sending the command, monitor its progress using the {\em Auto-update} button on the main screen. Check
the log text-area for a {\bf CONFIG\_DONE} message. If its {\em successful} field is {\em true} the configuration
succeeded, otherwise the DONE message should contain an error message describing what went wrong. If it was
a problem with the {\bf FOCUS\_OFFSET} the CCS issued, see Section \ref{sec:issconfiguration} to configure
the ISS correctly.

\subsection{Instrument Configuration List}
\label{sec:instrumentconfigurationlist}

When the CONFIG dialog's {\em Config} button is pressed, the list shown in Figure \ref{fig:icsguiconfiglistdialog} 
comes up.

\setlength{\unitlength}{1in}
\begin{figure}[!h]
	\begin{center}
		\begin{picture}(2.0,2.0)(0.0,0.0)
			\put(0,0){\special{psfile=icsgui_config_list_dialog.eps hscale=50 vscale=50}}
		\end{picture}
	\end{center}
	\caption{\em The IcsGUI Instrument Configuration List Dialog.}
	\label{fig:icsguiconfiglistdialog} 
\end{figure}

There is a list of configurations to select from, and also a {\em File} and {\em Filter} menu.
The basic mode of operation is to scroll down the list, select (left click) the configuration required, and
then select the {\em File$\rightarrow$Select} menu option which selects the configuration, quits the list dialog, 
and put the selected configuration in the CONFIG Dialog ready to send to the CCS.

If you decide not to select a configuration, select the {\em File$\rightarrow$Quit} menu option.

If you want more details of what a configuration will do, select the configuration in the list and use the
{\em File$\rightarrow$Amend} menu option to bring up the Add/Amend Configuration Dialog for the selected
configuration (See Section \ref{sec:addamendconfiguration}). 
Note with this dialog you can change the configuration as well!

You can add configurations using the {\em File$\rightarrow$Add} menu to add a configuration for the relevant
instrument. See Section \ref{sec:addamendconfiguration} for details on the dialog it brings up.

Configurations can be deleted using the {\em File$\rightarrow$Delete} menu option. Make sure you really want to do this!

The {\em Filter} menu allows you to filter what is displayed in the list. Currently configurations in the list
are for all instruments, but you can filter them to be just for the instrument you are talking to.

\subsection{Add/Amend Configuration}
\label{sec:addamendconfiguration}

Figure \ref{fig:icsguiconfigaadialog} shows the Add/Amend Configuration dialog box. 
Note this dialog is for RATCAM/FARTCAM imager 
configurations, other instruments have different Add/Amend dialogs.

\setlength{\unitlength}{1in}
\begin{figure}[!h]
	\begin{center}
		\begin{picture}(2.0,3.0)(0.0,0.0)
			\put(0,0){\special{psfile=icsgui_config_aa_dialog.eps hscale=50 vscale=50}}
		\end{picture}
	\end{center}
	\caption{\em The IcsGUI Instrument Configuration Add/Amend Dialog.}
	\label{fig:icsguiconfigaadialog} 
\end{figure}

This dialog allows you to change anything associated with the configuration.

The {\em Name} field must not be blank.

{\em Calibrate Before} and {\em Calibrate After} select whether to do any calibration before and subsequent
imaging operations. For imagers selecting them does nothing, other instruments may do something.

The {\em Lower Filter Wheel} and {\em Upper Filter Wheel} allows you to select which filters to use in each wheel.
You must type in the {\em Type Name} of the filter wanted, and ensure it is present in the correct wheel. Type
{\em clear} to have no filter in that wheel (assuming this is possible!).

The {\em X Binning} and {\em Y Binning} fields allow you to select how much on chip binning to do during readout.
Currently, they can only be set between 1 and 4, and should both have the same value (square binning only).

You can have up to 4 windows, rather than reading out the whole array. You must use {\em X Binning} and {\em Y Binning}
of {\em 1} to do this. Select the {\em Use} toggle to enable that window, and type in coordinates. If using more than
one window, you must ensure the windows do not overlap either horizontally or vertically.

\section{Taking an exposure}
\label{sec:exposing}

Before taking an image with the instrument, it must first be configured. See Section \ref{sec:configuring}.

Taking an exposure is normally done using the robotic command {\bf MULTRUN}. To bring up the MULTRUN Dialog, 
select the menu option {\em Send$\rightarrow$Expose$\rightarrow$Multrun}, see Figure \ref{fig:icsguiexposemenu}.

\setlength{\unitlength}{1in}
\begin{figure}[!h]
	\begin{center}
		\begin{picture}(3.0,3.0)(0.0,0.0)
			\put(0,0){\special{psfile=icsgui_expose_menu.eps hscale=50 vscale=50}}
		\end{picture}
	\end{center}
	\caption{\em The IcsGUI expose menu.}
	\label{fig:icsguiexposemenu} 
\end{figure}

\subsection{MULTRUN Dialog}

This brings up the MULTRUN Dialog, see Figure \ref{fig:icsguimultrundialog}.

\setlength{\unitlength}{1in}
\begin{figure}[!h]
	\begin{center}
		\begin{picture}(2.0,1.0)(0.0,0.0)
			\put(0,0){\special{psfile=icsgui_multrun_dialog.eps hscale=50 vscale=50}}
		\end{picture}
	\end{center}
	\caption{\em The IcsGUI Multrun Dialog.}
	\label{fig:icsguimultrundialog} 
\end{figure}

The Identifier field is automatically filled in by the IcsGUI, it allows the IcsGUI and CCS to identify
specific commands.

The {\em Standard} check-box, if selected, marks this exposure as being of an astronomical photometric standard field, 
and causes the CCS to save the image is a slightly different filename with a different OBSTYPE in the FITS header.

If the {\em Pipeline Process} check-box is selected, when the image(s) are taken they will be sent through the 
real time data pipeline, to be de-biased, flat-fielded and some measurements will be sent back in the DONE message
(e.g. FWHM).

The {\em Exposure Time} field must be filled in with the length the shutter should be open for. The length should
be in milliseconds.

The {\em No. of Exposures} field must be filled in with how many frames of the source you want to take. The number
must be greater than zero!

The {\em Ok} button sends the {\bf MULTRUN} command to the CCS.

The {\em Cancel} button quits the dialog without sending the {\bf MULTRUN} command.

After sending the command, monitor its progress using the {\em Auto-update} button on the main screen. 
During the MULTRUN, you may receive ISS requests for {\bf MOVE\_FOLD}, {\bf AUTOGUIDER\_ON}, {\bf AUTOGUIDER\_OFF},
{\bf OFFSET\_FOCUS} etc. Once the first image has been saved, you will receive a {\bf MULTRUN\_ACK} containing
the filename, which you can then view using the {\em FITS Filename} button. 

Check the log text-area for a {\bf MULTRUN\_DONE} message. If its {\em successful} field is {\em true} the 
{\bf MULTRUN} succeeded, otherwise the DONE message should contain an error message describing what went wrong. 
If it was a problem with an ISS command the CCS issued, see Section \ref{sec:issconfiguration} to configure
the ISS correctly.

\section{Bias/Darks etc}

The {\bf BIAS} and {\bf DARK} commands are both examples of {\bf CALIBRATE} commands. 
Before doing a {\bf BIAS} or {\bf DARK}, the instrument must first be configured. See Section \ref{sec:configuring}.

\subsection{BIAS Dialog}

The BIAS Dialog is selected from the Menu as follows: {\em Send$\rightarrow$Calibrate$\rightarrow$Bias}.
This brings up the BIAS Dialog, see Figure \ref{fig:icsguibiasdialog}.

\setlength{\unitlength}{1in}
\begin{figure}[!h]
	\begin{center}
		\begin{picture}(2.0,1.0)(0.0,0.0)
			\put(0,0){\special{psfile=icsgui_bias_dialog.eps hscale=50 vscale=50}}
		\end{picture}
	\end{center}
	\caption{\em The IcsGUI Bias Dialog.}
	\label{fig:icsguibiasdialog} 
\end{figure}

Operation of the BIAS Dialog is the same as for the MULTRUN Dialog, but with no fields to fill in! 
See Section \ref{sec:exposing} for details.

\subsection{DARK Dialog}

The DARK Dialog is selected from the Menu as follows: {\em Send$\rightarrow$Calibrate$\rightarrow$Dark}.
This brings up the DARK Dialog, see Figure \ref{fig:icsguidarkdialog}.

\setlength{\unitlength}{1in}
\begin{figure}[!h]
	\begin{center}
		\begin{picture}(2.0,1.0)(0.0,0.0)
			\put(0,0){\special{psfile=icsgui_dark_dialog.eps hscale=50 vscale=50}}
		\end{picture}
	\end{center}
	\caption{\em The IcsGUI Dark Dialog.}
	\label{fig:icsguidarkdialog} 
\end{figure}

Operation of the DARK Dialog is the same as for the MULTRUN Dialog, the {\em Exposure Time} field is again
in milliseconds. See Section \ref{sec:exposing} for details.

\section{Using the Reboot command}
\label{sec:rebootcommand}

The REBOOT Dialog is selected from the Menu as follows: {\em Send$\rightarrow$Interrupt$\rightarrow$Reboot}.
This brings up the REBOOT Dialog, see Figure \ref{fig:icsguirebootdialog}.

\setlength{\unitlength}{1in}
\begin{figure}[!h]
	\begin{center}
		\begin{picture}(2.0,1.0)(0.0,0.0)
			\put(0,0){\special{psfile=icsgui_reboot_dialog.eps hscale=50 vscale=50}}
		\end{picture}
	\end{center}
	\caption{\em The IcsGUI Reboot Dialog.}
	\label{fig:icsguirebootdialog} 
\end{figure}

The {\em Level} field must be filled in to determine what sort of REBOOT you wish to do.
The following lists the acceptable values:
\begin{enumerate}
\item {\em Re-datum}. This causes the CCS to re-read some of its configuration files (NOT the network ones),
	and restart the SDSU controller.
\item {\em Software}. This causes the CCS and DpRt processes to be killed. The Autobooter then kicks in
	to restart these processes, which causes them to re-read all their configuration, and re-datum the hardware.
\item {\em Hardware}. This causes the CCS to shut down its connection to the SDSU controller, and then issue
	a {\bf reboot} to the instrument control computer. After the control computer reboots, the Autobooter
	will ensure the CCS and DpRt start again.
\item {\em Power-Down}. This causes the CCS to shut down its connection to the SDSU controller, and then issue
	a {\bf shutdown} to the instrument control computer. The instrument control computer will then power down,
	or in the Solaris case will go to an {\em ok} prompt where it is safe to remove power. The SDSU controller
	and Filter Wheel Power can then also be powered down.
\end{enumerate}

The {\em Ok} button sends the {\bf REBOOT} command to the CCS.

The {\em Cancel} button quits the dialog without sending the {\bf REBOOT} command.

You may or may not get a {\bf REBOOT\_DONE} message appearing in the log text area, depending on the
type of REBOOT performed. If doing a level {\em 1} or {\em 2} REBOOT, use the {\em Auto-update} check-box
to determine when the CCS starts returning status again (when the CCS has restarted, this can take a minute or so,
it will return {\em Connection refused} errors in the meantime). 
Otherwise, the desktop will disappear and the machine will either shut down or reboot. 
Save any work you are doing first!

\section{Power off Sequence}
\subsection{With the CCD Control System running}
If the CCD Control System is running, power off as follows:
\begin{itemize}
\item Log in as {\bf eng} and start the IcsGUI.
\item Bring up the Reboot dialog ({\em Send$\rightarrow$Interrupt$\rightarrow$Reboot}). 
	See Section \ref{sec:rebootcommand}.
\item Set the level to {\bf 4}.
\item Press OK. The machine will now cleanly close connections to the SDSU controller, and
	shutdown the control computer (or get it into a state where the power can be removed).
	On a Solaris control computer, when the {\em ok} prompt appears it is safe to turn off the computers power.
	On a Linux control computer, the power will automatically turn off or the computer will print
	{\em Power off now.}.
\item Power off the control computer. See Section \ref{sec:diagramsunultra5back}.
\item Power off the filter wheel power supply.
\item Power off the SDSU controller power supply. See Section \ref{sec:diagramsdsupsu}.
\end{itemize}

\subsection{Without the CCD Control System running}
\begin{itemize}
\item Log in as {\bf root}.
\item Shut the computer down. On Solaris, type {\em shutdown -i 0 -y -g 1}. On Linux, type {\em shutdown -h now}.
\item Power off the control computer. See Section \ref{sec:diagramsunultra5back}.
\item Power off the filter wheel power supply.
\item Power off the SDSU controller power supply. See Section \ref{sec:diagramsdsupsu}.
\end{itemize}

\section{Diagrams}

\newpage
\subsection{PCI card}

\label{sec:diagrampcicard}
\setlength{\unitlength}{1in}
\begin{figure}[!h]
	\begin{center}
		\begin{picture}(5.0,5.0)(0.0,0.0)
			\put(0,0){\special{psfile=pci_card_low_res_annotated.eps hscale=70 vscale=70}}
		\end{picture}
	\end{center}
	\caption{\em SDSU PCI Card.}
	\label{fig:diagrampcicard} 
\end{figure}

\begin{enumerate}
\item Green light, if on PCI fibre optics are talking to controller fibre optics.
\item Little white button, used to reset PCI card in case of controller lock-ups.
\item Fibre optic receiver connector (with dust cap on).
\item Fibre optic transmitter connector (with dust cap on).
\end{enumerate}

\newpage
\subsection{SDSU Power supply}

\label{sec:diagramsdsupsu}
\setlength{\unitlength}{1in}
\begin{figure}[!h]
	\begin{center}
		\begin{picture}(5.0,5.0)(0.0,0.0)
			\put(0,0){\special{psfile=sdsu_psu_low_res_annotated.eps hscale=70 vscale=70}}
		\end{picture}
	\end{center}
	\caption{\em SDSU Power Supply.}
	\label{fig:diagramsdsupsu} 
\end{figure}

\begin{enumerate}
\item On/Off switch.
\item Reset button, resets timing and utility board.
\item Connector supplying PSU voltages to SDSU electronics box on dewar.
\end{enumerate}

\newpage
\subsection{Sun SPARC Ultra 5}

\label{sec:diagramsunultra5back}
\setlength{\unitlength}{1in}
\begin{figure}[!h]
	\begin{center}
		\begin{picture}(5.0,5.0)(0.0,0.0)
			\put(0,0){\special{psfile=sun_ultra5_back_low_res_annotated.eps hscale=70 vscale=70}}
		\end{picture}
	\end{center}
	\caption{\em Sun Ultra 5 Back.}
	\label{fig:diagramsunultra5back} 
\end{figure}

\begin{enumerate}
\item On/Off switch.
\item SDSU PCI Card normally here. See Section \ref{sec:diagrampcicard}.
\end{enumerate}

Do not use the On/Off switch on the front.

\newpage
\subsection{Dewar/Electronics Box}

\label{sec:diagramdewarelectronicsbox}
\setlength{\unitlength}{1in}
\begin{figure}[!h]
	\begin{center}
		\begin{picture}(4.0,6.0)(0.0,0.0)
			\put(0,0){\special{psfile=sdsu_dewar_electronics_back_low_res_annotated.eps hscale=70 vscale=70}}
		\end{picture}
	\end{center}
	\caption{\em Back of the dewar and electronics box.}
	\label{fig:diagramdewarelectronicsback} 
\end{figure}

\begin{enumerate}
\item Vacuum pump connector.
\item Cryotiger gas pipe - supply.
\item Cryotiger gas pipe - return.
\item SDSU power supply connector.
\item Power Control Board LED window.
\item Fibre Optic connector.
\item Fibre Optic connector.
\item Shutter cable connector.
\end{enumerate}

\section{Conventions/Abbreviations}
\begin{itemize}
\item {\bf ARI} - Astrophysics Research Institute.
\item {\bf CCD} - Charge Coupled Device. The first detector used by the telescope.
\item {\bf CCS} - CCD Control System. This is the camera control process.
\item {\bf DpRt} - The Data Pipeline Real Time module.
\item {\bf FARTCAM} - Faulkes Astronomical Robotic Telescope Camera. The FT imager instrument.
\item {\bf FITS} - Flexible Image Transport System. File format used to save astronomical images.
\item {\bf FT} - The Faulkes Telescope. Robotic 2m Telescope sited in Hawaii.
\item {\bf FWHM} - Full Width Half Maximum. A measure applied to FITS data, to determine, for instance,
	how good the {\em seeing} is.
\item {\bf GUI} - Graphical User Interface. A program with buttons, menus and windows.
\item {\bf ICC} - Instrument Control Computer. The computer on which the CCS runs. Also known as ISC.
\item {\bf IcsGUI} - Instrument Control System Graphical User Interface. The GUI used to control the CCS.
\item {\bf ISC} - Instrument Support Computer. The computer on which the CCS runs. Also known as ICC.
\item {\bf LT} - The Liverpool Telescope. Robotic 2m Telescope sited in La Palma.
\item {\bf PCI} - Peripheral Component Interface. The type of hardware interface used to talk to the SDSU controller.
\item {\bf RATCAM} - Robotic Astronomical Telescope Camera. The LT imager instrument.
\item {\bf RCS} - Robotic Control System. The program that usually controls both telescope, enclosure and
	all instruments.
\item {\bf SDSU} - San Diego State University.
\item {\bf TCS} - Telescope Control System. System which does high level control of the telescope.
\end{itemize}
 
\begin{thebibliography}{99}
\addcontentsline{toc}{section}{Bibliography}

\bibitem{bib:ccs}{\bf CCD Control System}
Liverpool John Moores University 
\newline{\em http://ltccd1.livjm.ac.uk/\verb'~'dev/ccs/latex/ccs.ps}

\bibitem{bib:ccs_aug}{\bf CCD Control System : Advanced User Guide}
Liverpool John Moores University 
\newline{\em http://ltccd1.livjm.ac.uk/\verb'~'dev/ccs/latex/ccs\_advanced\_user\_guide.ps}

\end{thebibliography}

\end{document}
