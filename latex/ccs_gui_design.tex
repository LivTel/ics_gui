\documentclass[10pt,a4paper]{article}
\pagestyle{plain}
\textwidth 16cm
\textheight 21cm
\oddsidemargin -0.5cm
\topmargin 0cm

\title{CCD Control System Interface Design}
\author{C. J. Mottram}
\date{}
\begin{document}
\pagenumbering{arabic}
\thispagestyle{empty}
\maketitle
\begin{abstract}
This document describes the design of the CCD Control System Interface. This is a process
used to control the CCD Control System interactively so that we can test it.
\end{abstract}

\centerline{\Large History}
\begin{center}
\begin{tabular}{|l|l|l|p{15em}|}
\hline
{\bf Version} & {\bf Author} & {\bf Date} & {\bf Notes} \\
\hline
0.1 &              C. J. Mottram & 18/11/99 & First draft \\
\hline
\end{tabular}
\end{center}

\newpage
\tableofcontents
\listoffigures
\listoftables
\newpage

\section{Introduction}
The process that is in control of the camera electronics on the Liverpool Telescope is called the
CCD Control System (CCS). It receives commands from the Robotic Control System (RCS) (via the 
Instrument Support System (ISS)) that enable it to control the camera electronics.

One of the requirements for the CCS is that an interface is provided to
{\em emulate} the ISS so that the ISS to CCS command set can be sent interactivly to the CCS.
This enables the CCS to be tested without having the rest of the telescope control systems enabled.

\subsection{Related Documentation}
All the requirements for the interface are derived from the {\em Liverpool Telescope: CCD
Camera Requirements} \cite{bib:ccdr}.

The command set the ISS uses to communicate with the CCS is part of the {\em ngat.message} package.
Documentation for this is at \cite{bib:messissinst} and the generated message classes are
described at \cite{bib:messageclasses}.

The documentation for the prototype CCS is located at \cite{bib:ccstop}. Communication between processes
is performed using the Java Message System for communication. An overview of this can be found at
\cite{bib:ngatnet}.

\section{Requirements}
All the requirements for the interface are derived from the {\em Liverpool Telescope: CCD
Camera Requirements} \cite{bib:ccdr}. The most important are in the {\em 28. Local User Control} section.
Section {\em 28.1 User Interface} states that a graphical user interface is required to control the
CCD and associated hardware, which is this interface. It then goes on to outline what the interface
requires, as shown in Tables \ref{tab:requirements1} and \ref{tab:requirements2}.

\begin{table}[!h]
\begin{center}
\begin{tabular}{|l|l|p{19em}|}
\hline
{\bf Requirement} & {\bf Name} & {\bf Description} \\ \hline
CCDR28.2	& Exposure Time & Settable to a presicision of 0.1 seconds in the range 0.1 to 1000 seconds.\\ \hline
CCDR28.3.1	& Independant Wheel Control & We need to be able to set the position of each filter wheel using 
		a physical position.\\ \hline
CCDR28.3.2	& Abstracted Filter Control & We need to be able to set the position of each filter wheel 
		using a string description. \\ \hline
CCDR28.4	& Repeated Exposure Configuration & We need to be able to do several exposures with the same
		configuration. \\ \hline
CCDR28.5	& Readout Speed Configuration & We need to be able to vary the readout speed between several
		levels. \\ \hline
CCDR28.6	& Chip Window and Binning Configuration & The ability to set up in advance of an exposure 
		the windowing and binning parameters
		for the chip should be provided.  This should be provided via a
		system of user configuration files which are editable via a graphic
		representation of a user-selected image taken from the temporary data store.\\ \hline
CCDR28.7	& Chip Window and Binning Selection & The ability to select the Chip Window and Binning 
		configuration file to be used for a given Single or Multiple Exposure should be provided.\\ \hline
CCDR28.8	& Exposure Control & The ability to start, stop and suspend exposures and groups of exposures
		should be provided.\\ \hline
CCDR28.8.1	& Aborting Exposures & It should be possible to interrupt a single or multiple exposure of 
		the CCD at any time during clearing, 
		exposure or read-out and for the system to be ready to obtain the
		next exposure in less than 20 seconds. \\ \hline
\end{tabular}
\end{center}
\caption{\em Requirements.}
\label{tab:requirements1}
\end{table}

\begin{table}[!h]
\begin{center}
\begin{tabular}{|l|l|p{25em}|}
\hline
{\bf Requirement} & {\bf Name} & {\bf Description} \\ \hline
CCDR28.9	& Status Display & A status display should display at least the following instrument characteristics:
		\begin{enumerate}
		\item CCD Status (CLEARING/ EXPOSING/ READING\_OUT/ IDLE/ FAULT etc).
		\item Current Binning and Windowing Config.
		\item Requested Exposure Time.
		\item Remaining Exposure Time.
		\item Remaining Number of Exposures of a multiple-exposure run.
		\item Filter wheel positions and current filter combination selected.
		\item CCD Temperature.
		\item Object Name as will be written to the FITS file.
		\item A scrollable text box for fault reporting.
		\end{enumerate}
		\\ \hline
CCDR29.1	& Interface Requirement & We require that a suitable interface is defined to allow
		the functionality defined in CCDR Section 28 
		to be employed by a suitably authorized user at a remote site.\\ \hline
CCDR32.1	& Control Interface & An engineering control interface that is accessible locally 
		and remotely should be provided.\\ \hline
CCDR32.2.1	& Status Monitoring & Access to the engineering interface for collection of Status and
		Diagnostic information should be possible without authorization.\\ \hline
CCDR32.2.2	& System Adjustments & Access to the engineering interface in order to make
		adjustments to any of the system parameters should be only granted to
		suitably authorized users at suitably authorized sites.\\ \hline
CCDR32.3	& Bias and Clock Voltages & The facility to alter the bias and clock voltages supplied 
		to the detector should be provided by the engineering interface.\\ \hline
CCDR32.4	& Clock Timing & The facility to alter the timing of the clock pulses should be provided by the
		engineering interface.\\ \hline
CCDR32.5	& Detector Temperature &The facility to monitor and alter the detector temperature should be provided 
		by the engineering interface.
		The temperature should be monitored to an accuracy of 1$^\circ$ C.\\ \hline
CCDR32.6	& Filter Wheel Configuration & The ability to alter the mapping of the filter wheel contents to 
		filter wheel positions should be provided by the engineering interface.\\ \hline
\end{tabular}
\end{center}
\caption{\em Requirements (continued).}
\label{tab:requirements2}
\end{table}

Most of the user requirements will be able to be met with this interface. Those that cannot will
be met using alternative methods. Configuration options that are not controlable by the user are currently
stored in configuration files that are read by the CCS. The requirements that will not be met with the
 interface are shown in Table \ref{tab:nireq}.

\begin{table}[!h]
\begin{center}
\begin{tabular}{|l|l|p{25em}|}
\hline
{\bf Requirement} & {\bf Name} & {\bf Problem/Workaround} \\ \hline
CCDR28.3.1	& Independant Wheel Control & Currently there is nothing in the command set to enable us to set
		filter wheel positions. This has to be set using Filter descriptions instead, as we are trying to
		abstract away from physical positions of filters.\\ \hline
CCDR28.5	& Readout Speed Configuration & This configuration is held in the CCS configuration file, and can be
		edited by an authorised user.\\ \hline
CCDR32.2.2	& System Adjustments & These parameters are modified by editing configuration files.\\ \hline
CCDR32.3	& Bias and Clock Voltages & 
		These values are currently set in the DSP code downloaded to the SDSU CCD Controller.\\ \hline
CCDR32.4	& Clock Timing & 
		These values are currently set in the DSP code downloaded to the SDSU CCD Controller.\\ \hline
CCDR32.6	& Filter Wheel Configuration & The ability to alter the mapping of the filter wheel contents to 
		filter wheel positions is done by editing a configuration file.\\ \hline
\end{tabular}
\end{center}
\caption{\em Requirements not in the interface.}
\label{tab:nireq}
\end{table}

Most of these requirements are only needed for the engineering interface. The current thinking is that the
{\em voodoo} program supplied by SDSU will form the basis of the engineering interface. The 
Readout Speed Configuration and System Adjustments can be done using {\em voodoo}. The Bias and Clock Voltages
can be changed using the SBV SDSU DSP comamnd, calls to this need putting in {\em libccd} and in the CCS
configuration file. {\em voodoo} can modify these numbers by typing in the SBN and SBV command manually.

The Clock Timing configuration can only be edited by changing the DSP code that is downloaded to the
controller. SDSU supply a tool to help you do this called {\em wvp} (Waveform Visualisation Program)
that provides a graphical interface.

\section{Communication}
The CCS interface will communicate with the SDSU CCD Controller electronics via the CCS process
itself. This will allow us to test the CCS as well as the underlying electronics. SDSU will be supplying
{\em voodoo}, which has a user interface but which communicates with the camera electronics directly. This will
allow us to use the SDSU CCD Controller independantly of the CCS if required.

This basically means the interface will be communicating with the CCS rather than the camera electronics
directly. It can use the ISS command set to communicate with the CCS. This means the interface will
be written in Java, to make use of the {\em ngat.message} message classes and the {\em ngat.net} network
communication classes.

\section{User Interface}
The user interface will be written in Java, using the Swing widget set.

\subsection{Main Screen}
The main screen will contain graphical elements showing the current status of the camera electronics.
The precise data shown is given in CCDR28.9. They will be a mechanism for updating this information
(an {\em update} button) possibly with extra graphical elements giving the option of an auto-update every
{\em n} seconds. There will be a log area, with the returned status from the commands being run being displayed. 
This may be replaced by a single Log Window program, so all GUI's can log to the same log window.
There will be a menu system allowing the user to quit the program, and also to bring up
a dialog to invoke a command.

Figure \ref{fig:uimainscreen} shows the layout of the main screen.
\setlength{\unitlength}{1in}
\begin{figure}[!h]
	\begin{center}
		\begin{picture}(8.0,3.1)(0.0,0.0)
			\put(0,0){\special{psfile=ccs_gui_ui_main.ps   hscale=70 vscale=75}}
		\end{picture}
	\end{center}
	\caption{\em User interface main screen layout.}
	\label{fig:uimainscreen} 
\end{figure}

An example of the menu structure is shown in Figure \ref{fig:uimenu}.

\setlength{\unitlength}{1in}
\begin{figure}[!h]
	\begin{center}
		\begin{picture}(8.0,3.1)(0.0,0.0)
			\put(0,0){\special{psfile=ccs_gui_ui_menu.ps   hscale=70 vscale=75}}
		\end{picture}
	\end{center}
	\caption{\em User interface menu layout.}
	\label{fig:uimenu} 
\end{figure}

\subsection{Command Dialogs}
There shall be a separate dialog for each command to be invoked. Only one of these dialogs should be
display-able at once (except for dialogs for INTERRUPT commands). Each of these dialogs will contain fields
suitable for inputing the data that makes up the parameters for each command. Hopefully, it may be possible to
write a program to automatically generate the dialog source files from the command source files. This will enable
us to keep the dialogs up to date with respect to the ISS command set easier.

An example dialog is presented in Figure \ref{fig:uirunatscreen}.
\setlength{\unitlength}{1in}
\begin{figure}[!h]
	\begin{center}
		\begin{picture}(8.0,3.1)(0.0,0.0)
			\put(0,0){\special{psfile=ccs_gui_ui_runat.ps   hscale=70 vscale=75}}
		\end{picture}
	\end{center}
	\caption{\em User interface Runat Dialog layout.}
	\label{fig:uirunatscreen} 
\end{figure}

The command dialogs should verify their input before proceeding to send the command. If input
verification fails, the input dialog should remain managed until the relevant field has a correct
value and the {\em Ok} or {\em Cancel} button has been pressed. The GUI should display a helpful error
message, as seen in Figure \ref{fig:uierrorscreen}.
\setlength{\unitlength}{1in}
\begin{figure}[!h]
	\begin{center}
		\begin{picture}(8.0,3.1)(0.0,0.0)
			\put(0,0){\special{psfile=ccs_gui_ui_error.ps   hscale=70 vscale=75}}
		\end{picture}
	\end{center}
	\caption{\em User interface error dialog layout.}
	\label{fig:uierrorscreen} 
\end{figure}

\subsection{FITS image display}
The interface will also be able to display the FITS data saved to disc. This will only work
when the interface is running on the local machine where the FITS images are stored. The
interface will use an external image viewer such as {\em ximtool}, which contains a programatical interface for
sending images for it to display. There may be a configuration dialog associated with this communication
mechanism.

\section{Conventions/Abbreviations}
\begin{itemize}
\item API - Application Programming Interface - a series of functions/procedures that implements some 
functionality.
\item CCD - Charge Coupled Device. The first detector used by the telescope.
\item CCS - CCD Control System. This is the camera control process.
\item GUI - Graphical User Interface.
\item ISS - The Instrument Support System - the generic instrument control process.
\item SDSU - San Diego State University.
\end{itemize}

\begin{thebibliography}{99}
\addcontentsline{toc}{section}{Bibliography}
\bibitem{bib:ccdr}{\bf Liverpool Telescope: CCD Camera Requirements}
I. A. Steele, D. Carter. {\em http://ltccd1.livjm.ac.uk/\verb'~'dev/latex/ccdr.ps}
\bibitem{bib:messissinst}{\bf package ngat.message.ISS\_INST}
LJMU {\em http://ltccd1.livjm.ac.uk/\verb'~'dev/mess/ISS\_INST.html}
\bibitem{bib:messageclasses}{\bf package ngat.message Class Hierarchy}
LJMU {\em http://ltccd1.livjm.ac.uk/\verb'~'dev/mess/tree.html}
\bibitem{bib:ccstop}{\bf CCD Control System (CCS) Documentation}
LJMU {\em http://ltccd1.livjm.ac.uk/\verb'~'dev/ccs/index.html}
\bibitem{bib:ngatnet}{\bf Java Message System: ngat.net package}
C. J. Mottram {\em http://ltccd1.livjm.ac.uk/\verb'~'dev/ngat/latex/ngat.net.ps}
\end{thebibliography}

\end{document}
